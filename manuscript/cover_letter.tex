\documentclass[11pt]{article}
\usepackage[margin=1in]{geometry}
\usepackage{setspace}
\usepackage{hyperref}

\begin{document}
\singlespacing

% --- Header ---
\noindent \today\\

\vspace{0.5em}
\noindent Editor\\
\textit{Psychological Science}\\

\vspace{1em}
\noindent Dear Editor,\\

\vspace{0.75em}
\noindent Please consider my manuscript, \textbf{``Flat terrain predicts fraud,''} for publication in \textit{Psychological Science}.

\vspace{0.75em}
\noindent \textbf{The puzzle.}
Governments lost over \$100 billion to COVID-19 relief fraud despite centralized digital systems designed to prevent it. The puzzle is not that fraud occurred---it is that fraud clustered geographically even when rules and detection were uniform. What makes some places more prone to rule violation than others, independent of enforcement?

\vspace{0.75em}
\noindent \textbf{The finding.}
Using Japan's Business Continuity Support Grant---where all 4.4 million applications passed through a single central portal with no local enforcement involvement---I show that flat terrain predicts substantially more fraud. One standard deviation flatter within a prefecture, 23\% more fraud cases (IRR = 0.77, $p = .003$). The flattest municipalities show five times the fraud rate of the most rugged. The effect survives permutation tests ($p < 10^{-4}$), holds across all 47 leave-one-prefecture-out replications, and scales monotonically with terrain. It appears only where fraud opportunities exist (denser, economically active areas) and attenuates by 65\% when controlling for population density and elderly share---consistent with anonymity as the operative mechanism.

\vspace{0.75em}
\noindent \textbf{The contribution.}
This work extends socioecological theory from governance \textit{preferences} to governance \textit{consequences}. Prior work shows rugged terrain predicts stronger endorsement of hierarchical authority (Tsudaka et al., 2025, \textit{Current Research in Ecological and Social Psychology}). I show the same ecology predicts lower rates of rule violation---linking preferences and behavior through a shared mechanism: terrain fragments populations, increases local observability, and raises the expected costs of identity-based deception. When neighbors know what you do, fraud becomes difficult.

\vspace{0.75em}
\noindent \textbf{Generality.}
The pattern is not specific to one program or country. Convergent analyses show similar terrain gradients for conventional theft in Japan ($n = 871$ municipalities; IRR = 0.80--0.87) and crime in South Korea ($n = 260$ police stations; IRR = 0.68--0.79). Panel analyses (Japan: 18 years; Korea: 6 years) confirm temporal stability with null terrain $\times$ time interactions. Geography predicts compliance before, during, and after the pandemic.

\vspace{0.75em}
\noindent \textbf{Transparency.}
All data and analysis scripts are publicly available at \url{https://github.com/gentsudaka/terrain-fraud}. The study uses aggregated administrative data and does not involve human-subject interaction.

\vspace{0.75em}
\noindent This manuscript is not under consideration elsewhere.

\vspace{1.0em}
\noindent Sincerely,\\

\vspace{2.0em}
\noindent Gen Tsudaka\\
Department of Psychology, The New School for Social Research\\
Columbia Business School, Columbia University\\
\href{mailto:tsudaka@proton.me}{tsudaka@proton.me}

\end{document}
